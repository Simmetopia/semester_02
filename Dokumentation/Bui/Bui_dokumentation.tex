\documentclass[11pt]{article}
\usepackage{graphicx}
\usepackage{siunitx}
\usepackage{cleveref}
\usepackage{keyval,kvoptions,fancyvrb,float,ifthen,calc,ifplatform,pdftexcmds,etoolbox,lineno}
\usepackage[utf8]{inputenc}
\usepackage[danish]{babel}
\usepackage[left=25mm, right=25mm, top=25mm, bottom=25mm]{geometry}
\usepackage{amsthm}
\setlength\parindent{0pt}
\usepackage{dcolumn}
%\usepackage{minted}
\usepackage{setspace}
\usepackage{tabularx}
\usepackage{booktabs}
\usepackage{multirow}
\usepackage{caption}

\title{Dokumentation}

\begin{document}

\begin{titlepage}

\newcommand{\HRule}{\rule{\linewidth}{0.5mm}} % Defines a new command for the horizontal lines, change thickness here

\center % Center everything on the page
 
%----------------------------------------------------------------------------------
%	HEADING SECTIONS
%----------------------------------------------------------------------------------

\textsc{\LARGE Aarhus School of Engineering}\\[1.0cm] % Name of your university/college
\textsc{\Large 2.SEMESTERPROJEKT}\\[0.1cm]
\textsc{\large E2PRJ2}\\[0.3cm]
\textsc{\large gruppe 10}\\[0.3cm] % Minor heading such as course title

%----------------------------------------------------------------------------------
%	TITLE SECTION
%----------------------------------------------------------------------------------

\HRule \\[0.4cm]
{ \huge \bfseries Smart Morning System - SMS}\\[0.1cm] % Title of your document
\HRule \\[0.6cm]

%----------------------------------------------------------------------------------
%	DATE SECTION
%----------------------------------------------------------------------------------

{\large \today}\\[0.5cm] % Date, change the \today to a set date if you want to be precise
 
%----------------------------------------------------------------------------------
%	AUTHOR SECTION
%----------------------------------------------------------------------------------

\begin{minipage}[t]{0.4\textwidth}
\raggedright \large
\emph{Forfattere:}\\
\begin{tabular}[t]{@{}r@{ }l@{}}
	201511621 & \textbf{Christian Brandstrup Bondesen}\\
	201511621 & \textbf{Emil Celik}\\
	201408914 & \textbf{Marc Auphong Bui}\\
	2015xxxxx & \textbf{Rasmus Lund}\\
	201406253 & \textbf{Simon Egeberg}\\
  \end{tabular}
\end{minipage}
~
\begin{minipage}[t]{0.4\textwidth}
\raggedleft \large
\emph{Vejleder:} \\
\textbf{Kim Bjerge} % Supervisor's Name
\vfill
\end{minipage}\\[0.8cm]


%----------------------------------------------------------------------------------
%	LOGO SECTION
%----------------------------------------------------------------------------------
\includegraphics[scale=0.6]{projektillustration.png}\\[0.6cm]
\includegraphics[scale=0.25]{forsidelogo.jpg}\\[1cm] % Include a department/university logo - this will require the graphicx package
%----------------------------------------------------------------------------------
\vfill % Fill the rest of the page with whitespace

\end{titlepage}

\tableofcontents
\vfill
\pagebreak

\section{Indledning}
\vfill
\pagebreak

\section{Kravspecifikation}
%Tabeloversigt over hovedansvarsområder for projektdeltagerne
\vfill
\pagebreak

\section{Systemarkitektur}
%----------------------------------------------------------------------------------
%	Hardware-arkitektur
%----------------------------------------------------------------------------------
\subsection{Hardware-arkitektur}

%----------------------------------------------------------------------------------
%	Overordnet BDD
%----------------------------------------------------------------------------------

\begin{figure}[H]
\centering
\includegraphics[scale=0.6]{Bdd-sms.png}
\label{FIG: Full BDD}
\caption{Overordnet BDD for Smart Morning System}
\end{figure}


\begin{table}[H]
\centering
	\begin{tabular}{l|p{10cm}|l|l}
	
	\toprule[0.4mm]\midrule \multicolumn{4}{c}{\textbf{OVERORDNET SYSTEM}}\\
	\midrule[0.4mm] Bloknavn & Funktionsbeskrivelse & Signal & Kommentar\\ \midrule[0.3mm]
	X-10 Sender & Modtage data serielt og sende data over lysnettet & 18V AC & Lysnet\\
	 & & 5V DC & VCC\\
	 & & 0V & Stel\\
	 & & Lås & DE2\\
	 & & Signal & Serielt\\
	 \midrule
	 Kodelås & Sender højt eller lavt signal & Lås & DE2\\
	 & & 0V & Stel\\
	 \midrule
	 Wake-up Light & \multirow{2}{10cm}{Tænder/slukker til et vis tidspunkt relativt til modtaget data fra X-10 modtageren \vfill}  & Signal & Serielt\\
	 & & 0V & Stel\\
	 & & 5V DC & VCC\\
	 \midrule
	 Electronics & \multirow{2}{10cm}{Tænder/slukker til et vis tidspunkt relativt til modtaget data fra X-10 modtageren \vfill}  & Signal & Serielt\\
	 & & 0V & Stel\\
	 & & 5V DC & VCC\\
	 \midrule
	 X-10 Modtager & \multirow{2}{10cm}{Modtage data fra lysnet og sende videre til hhv. Wake-up Light og Electronics \vfill} & 18V AC & Lysnet\\
	 & & 5V DC & VCC\\
	 & & 0V & Stel\\
	 & & Signal & Serielt\\
	 \midrule\bottomrule[0.4mm]

	\end{tabular}
	\caption{Blokbeskrivelse for det overordnede system}
	\label{tab: Bloktabel}
\end{table}

\subsubsection{X-10 Sender Blokbeskrivelse}

%----------------------------------------------------------------------------------
%	Blokbeskrivelse X-10 Sender || ZeroX-sender
%----------------------------------------------------------------------------------


\begin{minipage}[Ht]{0.45\linewidth}
	\centering
	\includegraphics{ZeroX-sender-blok.png}
	%\captionof{figure}{ZeroX-sender Blok}\label{FIG: ZeroX sender blok}
\end{minipage}
\hfill
\begin{minipage}[!t]{0.45\linewidth}
	\centering
   Her skal der stå noget tekst om Zero-Cross Detectoren for senderen
\end{minipage}%
\hfill

\begin{table}[H]
\centering
	\begin{tabular}{l|l|l}
	
	\toprule[0.4mm]\midrule \multicolumn{3}{c}{\textbf{ZeroX-sender Blok}}\\
	\midrule[0.4mm] Navn & Funktionsbeskrivelse & Signal\\ \midrule[0.3mm]
	 Lysnet & Modtage signal fra lysnettet & 18V AC\\
	 Zc & Sender digital signal til Arduino & Digital\\
	 VCC & Spændingsforsyning til ZeroX-Detector & 5V DC\\
	 Stel & Fælles stel  & 0V\\
	 \midrule\bottomrule[0.4mm]

	\end{tabular}
	\caption{Blokbeskrivelse for senderens Zero Cross Detector}
	\label{tab: Bloktabel ZeroX sender}
\end{table}
\qquad

%----------------------------------------------------------------------------------
%	Blokbeskrivelse X-10 Sender || Generator-sender
%----------------------------------------------------------------------------------

\begin{minipage}[Ht]{0.45\linewidth}
	\centering
	\includegraphics{Generator-sender-blok.png}
	%\captionof{figure}{Generator-sender Blok}\label{FIG: Generator sender blok}
\end{minipage}
\hfill
\begin{minipage}[!t]{0.45\linewidth}
	\centering
   Her skal der stå noget tekst om 120kHz-carrier Generator for senderen
\end{minipage}%
\hfill

\begin{table}[H]
\centering
	\begin{tabular}{l|l|l}
	
	\toprule[0.4mm]\midrule \multicolumn{3}{c}{\textbf{Generator-sender Blok}}\\
	\midrule[0.4mm] Navn & Funktionsbeskrivelse & Signal\\ \midrule[0.3mm]
	 Tx & Modtage Burstsignal fra sender Arduino & Burst\\
	 VCC & Spændingsforsyning til Generatoren & 5V DC\\
	 Stel & Fælles stel  & 0V\\
 	 Data & Sender data signal til lysnettet & X-10 signal\\
	 \midrule\bottomrule[0.4mm]

	\end{tabular}
	\caption{Blokbeskrivelse for senderens Generator}
	\label{tab: Bloktabel Generator sender}
\end{table}
\qquad
%----------------------------------------------------------------------------------
%	Blokbeskrivelse X-10 Sender || Arduino-sender
%----------------------------------------------------------------------------------

\begin{minipage}[Ht]{0.45\linewidth}
	\centering
	\includegraphics{Arduino-sender-blok.png}
	%\captionof{figure}{Arduino-sender Blok}\label{FIG: Arduino sender blok}
\end{minipage}
\hfill
\begin{minipage}[!t]{0.45\linewidth}
	\centering
   Her skal der stå noget tekst om Arduinoen for senderen
\end{minipage}%
\hfill

\begin{table}[H]
\centering
	\begin{tabular}{l|l|l}
	
	\toprule[0.4mm]\midrule \multicolumn{3}{c}{\textbf{Arduino-sender Blok}}\\
	\midrule[0.4mm] Navn & Funktionsbeskrivelse & Signal\\ \midrule[0.3mm]
	 Zc & Modtage digitalt ZeroX signal fra ZeroX Detector & Digital\\
	 Stel & Fælles stel  & 0V\\
	 Tx & Sender 120kHz burst i perioder af 1 ms & Burst\\
	 \midrule\bottomrule[0.4mm]

	\end{tabular}
	\caption{Blokbeskrivelse for senderens Arduino}
	\label{tab: Bloktabel Arduino sender}
\end{table}
\qquad

\subsubsection{X-10 Modtager Blokbeskrivelse}

%----------------------------------------------------------------------------------
%	Blokbeskrivelse X-10 Modtager || ZeroX-modtager
%----------------------------------------------------------------------------------


\begin{minipage}[Ht]{0.45\linewidth}
	\centering
	\includegraphics{ZeroX-modtager-blok.png}
	%\captionof{figure}{ZeroX-sender Blok}\label{FIG: ZeroX sender blok}
\end{minipage}
\hfill
\begin{minipage}[!t]{0.45\linewidth}
	\centering
   Her skal der stå noget tekst om Zero-Cross Detectoren for modtageren
\end{minipage}%
\hfill

\begin{table}[H]
\centering
	\begin{tabular}{l|l|l}
	
	\toprule[0.4mm]\midrule \multicolumn{3}{c}{\textbf{ZeroX-modtager Blok}}\\
	\midrule[0.4mm] Navn & Funktionsbeskrivelse & Signal\\ \midrule[0.3mm]
	 Lysnet & Modtage signal fra lysnettet & 18V AC\\
	 Zc & Sender digital signal til Arduino & Digital\\
	 VCC & Spændingsforsyning til ZeroX-Detector & 5V DC\\
	 Stel & Fælles stel  & 0V\\
	 \midrule\bottomrule[0.4mm]

	\end{tabular}
	\caption{Blokbeskrivelse for modtagerens Zero Cross Detector}
	\label{tab: Bloktabel ZeroX modtager}
\end{table}
\qquad
%----------------------------------------------------------------------------------
%	Blokbeskrivelse X-10 Modtager || Detector-modtager
%----------------------------------------------------------------------------------

\begin{minipage}[Ht]{0.45\linewidth}
	\centering
	\includegraphics{Detector-modtager-blok.png}
	%\captionof{figure}{Generator-sender Blok}\label{FIG: Generator sender blok}
\end{minipage}
\hfill
\begin{minipage}[!t]{0.45\linewidth}
	\centering
   Her skal der stå noget tekst om 120kHz-carrier Detector for modtageren
\end{minipage}%
\hfill

\begin{table}[H]
\centering
	\begin{tabular}{l|l|l}
	
	\toprule[0.4mm]\midrule \multicolumn{3}{c}{\textbf{Detector-modtager Blok}}\\
	\midrule[0.4mm] Navn & Funktionsbeskrivelse & Signal\\ \midrule[0.3mm]
	 Lysnet & Modtage overlejret signal fra lysnettet & 18V AC\\
	 Data & Sender data signal til Arduino & X-10 signal\\
	 Stel & Fælles stel  & 0V\\
	 VCC & Spændingsforsyning til Generatoren & 5V DC\\
 	 
	 \midrule\bottomrule[0.4mm]

	\end{tabular}
	\caption{Blokbeskrivelse for modtagerens Detector}
	\label{tab: Bloktabel Detector modtager}
\end{table}
\qquad
%----------------------------------------------------------------------------------
%	Blokbeskrivelse X-10 Modtager || Arduino-modtager
%----------------------------------------------------------------------------------

\begin{minipage}[Ht]{0.45\linewidth}
	\centering
	\includegraphics{Arduino-modtager-blok.png}
	%\captionof{figure}{Arduino-sender Blok}\label{FIG: Arduino sender blok}
\end{minipage}
\hfill
\begin{minipage}[!t]{0.45\linewidth}
	\centering
   Her skal der stå noget tekst om Arduinoen for modtageren
\end{minipage}%
\hfill

\begin{table}[H]
\centering
	\begin{tabular}{l|l|l}
	
	\toprule[0.4mm]\midrule \multicolumn{3}{c}{\textbf{Arduino-modtager Blok}}\\
	\midrule[0.4mm] Navn & Funktionsbeskrivelse & Signal\\ \midrule[0.3mm]
	 Zc & Modtage digitalt ZeroX signal fra ZeroX Detector & Digital\\
	 Stel & Fælles stel  & 0V\\
	 Data & Modtage data signal fra Detektor & X-10 signal\\
	 Serielt & Sender signal til Wake-up Light og Electronics med X-10 data & Signal\\
	 \midrule\bottomrule[0.4mm]

	\end{tabular}
	\caption{Blokbeskrivelse for modtagerens Arduino}
	\label{tab: Bloktabel Arduino modtager}
\end{table}
\qquad


%----------------------------------------------------------------------------------
%	Software-arkitektur
%----------------------------------------------------------------------------------
\subsection{Software-arkitektur}

\vfill
\pagebreak

\section{Hardware-design, implementering \& modultest}
\subsection{Design (HW)}
\subsection{Implementering (HW)}
\subsection{Modultest (HW)}
\vfill
\pagebreak

\section{Software-design, implementering \& modultest}
\subsection{Design (SW)}
\subsection{Implementering (SW)}
\subsection{Modultest (SW)}
\vfill
\pagebreak

\section{Integrationstest (HW/SW)}
%Valgfrit på 2.semester
\vfill
\pagebreak

\section{Accepttest}
\vfill
\pagebreak

\section{Bilag}
\vfill
\pagebreak



\end{document}
